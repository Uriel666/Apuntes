\documentclass[12pt]{article}
\usepackage{amsmath}
\usepackage{amssymb}
\usepackage{amsthm}
\usepackage{enumerate}

\theoremstyle{definition}
\newtheorem{definition}{Definition}[section]


\setlength{\parskip}{\baselineskip}
\setlength{\parindent}{0pt}

\begin{document}
    \section{What is Logic ?}

    \subsection{What is an argument?}

    By argument ``argument'' we mean, roughly, a chain of reasoning in support of a certain
    conclusion. So we must distinguish arguments from mere disagreements and disputes.\par 

    \subsection{What sort of evaluation?}

    The business of logic, then, is the evaluation of stretches of reasoning. Let's take a vey simple 
    case (call it argument A).\par 

    (1) \textit{All philosophers are eccentric.}\par

    I then introduce you to Jack, who I tell you is a philosopher. So you come to believe\par

    (2) \textit{Jack is a philosopher}\par

    Puttinh these two thoughts togeether, you can obviously draw the conclusion \par 

    (3) Jack is eccentric\par 

    This little bit of reasoning cannow be evaluated along two quite independent dimensions.\par 

    \begin{itemize}
        \item First, we can ask whether the \textit{premises} (1) and (2) are true
        \item Second, we can ask about the qality of the \textit{inference} from the premisses (1) and (2) to the
        conclusion (3)  is surely absolutely comeplling. If (1) and (2) are granted to be true (granted ``for the sake of argument''
        , as we say), then (3) has to go to be true too. Someone who asserted that Jack is a philosopher, and that all philosophers are 
        eccentrinc, yet went on to deny that Jack is eccentric, would be implicitly contradicting himself.
    \end{itemize}

    In brief, it is one thing to consider whether an argument starts from true premisses; it is another thing entirely to consider whether 
    it moves on by reliable inferential steps. We normally want to start from true permisses and to reason by steps which will take us on to further truths. \par 

    The premisses (and conslusions) of arguments can be about all sorts of topics: their truth is usually no business of the logician, by contrast, is not the truth 
    of initial premisses but the way we argue from a given starting point. Logic is not about whether our premisses are true but about whether our inferences really do 
    support our conclusions once the premisses are granted. It is in this sense that logic is concerned with the `internal cogency' of our reasoning.\par

    \subsection{Deduction vs. induction}
    As a first shot definition, we will say 

    \begin{definition}
        An inferece step is \textit{deductively valid} just, if given that its premisses are true, then its conclusion is absolutely 
        guaranteed to be true as well.
    \end{definition}

    Equivalently, when an inference is deductively valid, we'll say that the premisses \textit{logically entail} the conclusion.\par 

    \begin{definition}
        The extrapolation from the past to the future, or more generally from old cases to new cases, is standarly called \textit{inductive}
    \end{definition}


\end{document}